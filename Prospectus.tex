\documentclass[12pt,a4paper]{article}
%\usepackage[latin1]{inputenc}
\usepackage{amsmath}
\usepackage{amsfonts}
\usepackage{amssymb}
\usepackage{graphicx}
\usepackage{hyperref}
\usepackage{color}
\usepackage[compat=1.1.0]{tikz-feynman}
\usepackage{comment}
\usepackage{chngcntr}
\usepackage{caption}
\usepackage{subcaption}
\usepackage{setspace}
\usepackage{lineno}
\usepackage{cite}
\usepackage[utf8]{inputenc}
\usepackage{notoccite}
\usepackage{comment}

\newcommand{\rulesep}{\unskip\ \vrule\ }
\newcommand{\zeronubb}{$0\nu \beta \beta$}
\newcommand{\twonubb}{$2\nu \beta \beta$}


\counterwithin{figure}{subsection}
\counterwithin{table}{subsection}
\author{Christopher Davis\thanks{\url{christopher.davis@yale.edu}}}
\title{Measurement of 2$\nu\nu\beta\beta$ in CUORE}
\begin{document}

\begin{titlepage}
\centering

{\scshape\LARGE Yale University \par}
\vspace{1 cm}
{\scshape\Large Thesis Prospectus \par}
\vspace{1.5 cm}
{\huge\bfseries Measurement of Two Neutrino Double Beta Decay in CUORE \par}
\vspace{2 cm}
{\Large\itshape Christopher Davis \par}
\vfill
supervised by \par
Dr. ~Reina \textsc{Maruyama}

\vfill
Thesis Committee \par
Dr. ~Karsten \textsc{Heeger} \par
Dr. ~Keith \textsc{Baker}

\vfill
{\large \today \par}
\end{titlepage}

\tableofcontents
\listoffigures

\doublespacing
\linenumbers

\section{Introduction}

\subsection{Standard Model}

The Standard Model of particle physics is the main explanation for physical phenomena and has correctly predicted the existence of multiple particles -- most recently the Higgs boson \cite{Aad:2012tfa}\cite{Chatrchyan:2012xdj}. The Standard Model is one of the most powerful theories in physics, able to correctly predict phenomena on both subatomic and cosmological scales (\color{blue}citation needed\color{black}).

In the Standard Model, there are fermionic particles: the quarks and leptons that constitute matter and bosonic particles: the gauge bosons and the higgs boson that mediate the fundamental forces and give particles mass, respectively. 

\subsubsection*{Neutrinos}

However, not all is well with the theory. For example, one type of particle in the Standard Model, the neutrino, has properties that are in conflict with the Standard Model. Neutrinos have been observed to oscillate between their generational counterparts: electron-, mu-, and tau-neutrinos \cite{PhysRevLett.20.1205}\cite{Hatakeyama:1998ea}\cite{Ahmad:2001an}. This phenomenon implies that the neutrinos have a non-zero mass, which is in strong tension with the Standard Model, which assumes that neutrinos are massless.

The Standard Model can be extended to include the masses for the neutrinos, but this opens up another question for how the neutrino mass comes in to the Standard Model Lagrangian. For the other fundamental fermionic particles, such as the electron that has a $\textrm{e}^-$ charge, the antiparticle will have the opposite charge, such as the positron that has a $\textrm{e}^+$ charge. This therefore means that the particle and antiparticle are distinct particles, called Dirac fermions. The neutrino is unique in this regard as it is the only electrically neutral fundamental fermion. This means the neutrino may be identical to its own antiparticle, which we call a Majorana fermion, and, more generally, may have both a Dirac and Majorana mass.



\begin{align}
\mathcal{L} = & \frac{1}{2}m_D(\bar{\psi}_L\psi_R+\bar{\psi}^c_L\psi^c_R)+m_L\bar{\psi}_L\psi^c_R+m_R+\bar{\psi}^c_L\psi_R +h.c. \\
\mathcal{L} = &\frac{1}{2} \begin{pmatrix}
\bar{\psi}_L,& \bar{\psi}^c_L \\
\end{pmatrix} \begin{pmatrix}
m_L & m_D \\
m_D & m_R \\
\end{pmatrix}
\begin{pmatrix}
\psi^c_R \\
\psi_R
\end{pmatrix}
\end{align}

\subsubsection*{Neutrino Mass}

For the neutrino to have mass, that mass has to come from somewhere (\color{blue}citation needed\color{black}). While the specific origin of the neutrino masses is unknown, and, in particular, why the mass of the neutrino is $10^6$ times lighter than the electron, the next lightest known fundamental particle. There are some compelling theories for this difference, however. One of the simplest methods is known as the Seesaw mechanism. This method is the simplest because it only extends the SM by adding a single right-handed singlet neutrino, $\nu_R$, for each of the flavor states.

The type I seesaw mechanism then introduces a mass matrix for the neutrinos of the type:
\begin{equation}
M =\begin{pmatrix}
0 & m_D \\
m_D & M_R
\end{pmatrix}
\end{equation}
In the case where $M_R \gg m_D$, the two neutrino masses become $M_R$ and $\frac{m_D^2}{M_R}$, which, if we assume that the Dirac mass is similar to the other leptons ($\approx 1~\textrm{MeV}$), yields a meV-scale neutrino for a heavy neutrino of mass $10^{15} \textrm{eV}$.

\subsection{Baryogenesis}

Another issue with the Standard Model is that it cannot correctly predict the matter-antimatter asymmetry present in the universe. After the Big Bang, matter and antimatter should have been produced in equal amounts. However, instead of being produced in equal numbers, the observable universe is made up entirely of only matter on large scales, implying that the universe has a preference for slightly more baryons than antibaryons (\color{blue}citation needed\color{black}). The requirements for a process to create this asymmetry are summed up in the Sakharov conditions for baryogenesis \cite{Sakharov:1967dj}:
\begin{enumerate}
\item Baryon number, B, violation
\item C and CP violation
\item Out of thermal equilibrium
\end{enumerate}
The first condition is needed so that there is a physics process within which a baryon can be created or an anti-baryon destroyed ($\Delta \textrm{B}\neq0$), the second condition requires that this process occurs more often for baryon creation and anti-baryon destruction than vice-versa ($\Gamma(\Delta \textrm{B}>1) > \Gamma(\Delta \textrm{B}<-1)$), and the third condition requires that the process has to occur when the universe is not in thermal equilibrium as other processes would have to oppose baryon-asymmetry in thermal equilibrium.

\subsection{Majorana Neutrinos and Leptogenesis}
One possible mechanism for baryogenesis is that the imbalance between matter and antimatter is due to an imbalance in leptons that transfers over to the baryons. This `leptogenesis' would have to fulfill the same Sakharov conditions except that there would need to be lepton number, L, violation. This violation is possible if the neutrino is its own antiparticle. In the interaction $d + d \rightarrow u + u + e + e$ shown below,  \hyperref[fig:0nuBB]{Fig \ref*{fig:0nuBB}}, a neutrino is exchanged between the two $W^-$ and allows for the final state to have two more leptons than the initial state, thereby violating lepton number conservation.


\begin{figure}[t!]
\centering
\begin{subfigure}[t]{0.49\textwidth}
\centering
\begin{tikzpicture}

\begin{feynman}
\vertex (a) {\(d\)};
\vertex [right=of a] (b);
\vertex [above right=of b] (f1) {\(u\)};
\vertex [below right = of b] (c);
\vertex [above right=of c] (f2) {\(e\)};
\vertex [below=of c] (f3);
\vertex [below left = of f3] (d);
\vertex [below right = of f3] (f4) {\(e\)};
\vertex [left = of d] (f5) {\(d\)};
\vertex [below right = of d] (f6) {\(u\)};

\diagram {
(a) -- [fermion] (b) -- [fermion] (f1),
(b) -- [boson, edge label'=\(W^{-}\)] (c),
(c) -- [fermion] (f2),
(f3) -- [plain, edge label'={\(\nu=\overline{\nu}\)}] (c),
(f3) -- [boson, edge label'=\(W^{-}\)] (d),
(f3) -- [fermion] (f4),
(f5) -- [fermion] (d) -- [fermion] (f6),
};
\end{feynman}
\end{tikzpicture}
\caption{}
\label{fig:0nuBB}
\end{subfigure}
\rulesep
\begin{subfigure}[t]{0.49\textwidth}
\centering
\begin{tikzpicture}

\begin{feynman}
\vertex (a) {\(d\)};
\vertex [right=of a] (b);
\vertex [above right=of b] (f1) {\(u\)};
\vertex [below right = of b] (c);
\vertex [above right=of c] (f2) {\(e\)};
\vertex [right = of c] (nu1) {\(\nu_e\)};
\vertex [below=of c] (d);
\vertex [below left = of d] (e);
\vertex [right = of d] (nu2) {\(\nu_e\)};
\vertex [below right = of d] (f4) {\(e\)};
\vertex [left = of e] (f5) {\(d\)};
\vertex [below right = of e] (f6) {\(u\)};

\diagram {
(a) -- [fermion] (b) -- [fermion] (f1),
(b) -- [boson, edge label'=\(W^{-}\)] (c),
(c) -- [fermion] (f2),
(d) -- [opacity = 0] (c),
(c) -- [anti fermion] (nu1),
(d) -- [boson, edge label'=\(W^{-}\)] (e),
(d) -- [fermion] (f4),
(d) -- [anti fermion] (nu2),
(f5) -- [fermion] (e) -- [fermion] (f6),
};
\end{feynman}
\end{tikzpicture}
\caption{}
\label{fig:2nuBB}
\end{subfigure}
\caption{(a) Diagram of Neutrinoless Double Beta Decay: $d+d\rightarrow u+u+e+e$. (b) Diagram of Two Neutrino Double Beta Decay: $d+d \rightarrow u+u+e+e+\nu+\nu$.}
\end{figure}

If this process exists, then, in the early universe, there could have been an imbalance in the number of leptons and antileptons that could have been transferred by sphaleron transitions to the baryons. Therefore, the current baryon asymmetry that is seen today might be due to lepton number violation, and a discovery could have profound implications on how we arrived at the universe we live in today.


\section{Double Beta Decay in the Laboratory}

In the laboratory, the fundamental interactions shown in \hyperref[fig:0nuBB]{Fig. \ref*{fig:0nuBB}} and \hyperref[fig:2nuBB]{Fig. \ref*{fig:2nuBB}} are measured as decays of specific nuclei. In particular, double beta decay occurs when two neutrons or two protons in a nucleus spontaneously decay, typically from the ground state of the initial nucleus (A,~Z) to the ground state of the final nucleus (A,~Z$\pm$2) (\color{blue}citation needed\color{black}). By energy conservation, the change in energy from the initial to the final nucleus, called the Q-value, is given in roughly equal amounts to the final state particles: the two electrons, and, in the case of \twonubb, the two neutrinos. The nuclear recoil is negligible in these decays as the nuclei involved are much more massive than the electrons and the neutrinos. 

When detecting double beta decay, the neutrinos pass undetected through the detector volume and only the electrons can be measured. Thus, the energies reconstructed by the electrons will be a continuous spectrum up to the Q-value for \twonubb~and a single peak at the Q-value for \zeronubb~\hyperref[fig:2nubbspectrum]{Fig. \ref*{fig:2nubbspectrum}}.

\begin{figure} [h]
\centering
\includegraphics[width=0.7\linewidth]{Figures/2nuBBSpectrum.png}
\caption{An example spectrum for double beta decay. The amplitude of the \zeronubb~decay is not shown to scale.}
\label{fig:2nubbspectrum}
\end{figure}


\subsection{\twonubb}
In general, it is difficult to measure double beta decay as it is a second-order weak process. The exception to this occurs in some even-even nuclei wherein single beta decay is energetically forbidden, but double beta decay is allowed. Of the 35 naturally occurring isotopes where double beta decay is possible, 12 of them have been measured in the laboratory. Of particular importance for this work is the double beta decay half-life for $^{130}\textrm{Te}$ at $0.7 \pm 0.09 \pm 0.11 \times 10^{21}~\textrm{years}$ as measured by the NEMO-3 collaboration (\color{blue}citation needed\color{black}). The half-lives for double beta decay are the longest currently measured, as the longest measured alpha decay is $^{209}\textrm{Bi}$ with a half-life of $1.9 \times 10^{19}~\textrm{years}$.

Since the half-life is so large, an experiment needs to have low background levels in order to be able to observe these events. This is generally realized in experiments by going to deep underground laboratories to escape cosmic radiation sources and by having pure and clean materials in and near the detector and sources. Also, due to the energies being produced according to a spectrum, all of the backgrounds with energy up to the Q-value will reduce the sensitivity of the measurement.

\begin{comment}
\begin{center}
\begin{tabular}{|c|c|c|}
\hline 
Isotope & Half-life, $10^{21}$ years & Experiment \\ \hline 
$^{48}\textrm{Ca}$ & $0.044^{+0.005}_{-0.004} \pm 0.004$ & Nemo-3 \\ \hline
$^{78}\textrm{Ge}$ & $1.84^{+0.09~+0.11}_{-0.08~-0.06}$ & GERDA (2013) \\ \hline
$^{82}\textrm{Se}$ & $0.096 \pm 0.003 \pm 0.010$ & NEMO-3 \\ \hline
$^{96}\textrm{Zr}$ & $0.0235 \pm 0.0014 \pm 0.0016$ & NEMO-3 \\ \hline
$^{100}\textrm{Mo}$ & $0.00711 \pm 0.00002 \pm 0.00054$ & NEMO-3 \\ \hline
$^{116}\textrm{Cd}$ & $0.028 \pm 0.001 \pm 0.003$ & NEMO-3 \\ \hline
%^{128}\textrm{Te}$ & $7200 \pm 400$ & Geochemical \\ \hline
$^{130}\textrm{Te}$ & $0.7 \pm 0.09\pm 0.11$ & NEMO-3 \\ \hline
$^{136}\textrm{Xe}$ & $2.165 \pm 0.016 \pm 0.059$ & EXO-200 \\ \hline
$^{150}\textrm{Nd}$ & $0.00911^{+0.00025}_{-0.00022}\pm 0.00063$ & NEMO-3 \\ \hline
%$^{238}\textrm{U}$ & $2.0 \pm 0.6$ & Radiochemical \\ \hline
\end{tabular} 
\end{center}
\end{comment}

\subsection{\zeronubb}
In principle, all nuclei that can undergo \twonubb~can also undergo \zeronubb. The main difference experimentally is that instead of searching for a spectrum, the signature of \zeronubb~is that of a single peak at the Q-value. This peak feature allows for experiments to be designed that are optimized to minimize backgrounds around the Q-value, which increases their sensitivity to \zeronubb. This is one of the main features of CUORE, although other techniques are used by other experiments, described in \hyperref[sec:zeronubb_Experiments]{Section \ref*{sec:zeronubb_Experiments}}.

Sensitivity to \zeronubb~ is generally given by a few basic parameters. Following \cite{Barea:2013bz}, the decay occurs according to

\begin{equation}
[\tau^{0\nu}_{1/2}]^{-1} = G_{0\nu}|M_{0\nu}|^2|f(m_i,U_{ei})|^2,
\end{equation}
where $G_{0\nu}$ can be considered as a kinematics term, $|M_{0\nu}|$ is the nuclear matrix element determined by the physics of the nucleus, and $|f(m_i,U_{ei})|$ is determined by the physics beyond the standard model including the masses $m_i$ and mixing matrix elements $U_{ei}$ of the neutrino species. This calculation is not a trivial one, however, as the nuclear matrix elements are require complex nuclear physics models. 

If we take the example approach to a \zeronubb~ search listed above, the experiment can be reduced to a counting experiment in a window of $\Delta E$ around the decay Q-value. This signal, $S$, will also have a corresponding background, $B$. In a given time, we could then expect 

\begin{equation}
S = \frac{\ln(2)\epsilon Nt}{T^{0\nu}_{1/2}}
\end{equation}
signal events and 
\begin{equation}
B = bMt\Delta E
\end{equation} 
background events, where $\epsilon$ is the signal detection efficiency, $N$ is the number of nuclei that can undergo \zeronubb, $t$ is the livetime, $b$ is the background rate per unit energy per unit detector mass, and $M$ is the total active mass of the detector. If we then consider the background to be externally sourced and obeying gaussian statistics, a 1$\sigma$ half-life sensitivity is then

\begin{equation}
T^{0\nu}_{1/2} = \ln(2)\frac{\epsilon a_I N_A \eta}{W} \sqrt{\frac{Mt}{b\Delta E}},
\end{equation}
where we have used the number of candidate nuclei $N = \frac{a_I N_A \eta}{W}M$ into the isotopic abundance $a_I$, the molar mass of the isotope $W$, and the number of nuclei of interest per molecule $\eta$ \cite{Alessandria:2011rc}. This breakdown, while adding complications, is useful as it allows the use of part of this expression as a figure of merit for an experiment's sensitivity to \zeronubb, given by

\begin{equation}
\textrm{Sensitivity} \propto a_I \sqrt{\frac{Mt}{b\Delta E}}
\end{equation}
Note that the sensitivity increases the same for a doubling of the isotopic abundance as it does by a quadrupling of the detector mass.

\subsection{Beta Decay Experiments} \label{sec:zeronubb_Experiments}

In addition to CUORE, many other experiments are currently searching and have searched for \zeronubb. Some of these experiments are listed below and the current status of the field is shown in \hyperref[fig:cuore-0-mbetabeta]{Fig \ref*{fig:cuore-0-mbetabeta}}.


\begin{figure}[htbp]
\centering
\includegraphics[width=0.7\linewidth]{"Figures/CUORE-0 mbetabeta"}
\caption{The current status of the field of \zeronubb. The previous experiments in the line of CUORE, CUORE-0 and Cuoricino, have their combined $^{130}$Te result and the results for Mo, Ge, and Xe are shown from NEMO-3, GERDA, and KamLAND-Zen, respectively.}
\label{fig:cuore-0-mbetabeta}
\end{figure}

\colorbox{red}{Many issues here. Need to define mbb, and sensitivity and hierarchies!}

\subsubsection*{GERDA}

The GERmanium Detector Array (GERDA) Experiment is currently searching for \zeronubb~ at the Laboratori Nationali del Gran Sasso (LNGS) using $^{76}$Ge as the source \cite{Agostini:2016iid}. The source is enriched to $\sim86\%$ and acts acts both the source and the detector of \zeronubb. GERDA takes advantage of some of the crystals having pulse-shape discrimination (PSD) between \zeronubb-like events and $\gamma$-like events in addition to scintillation light from the surrounding liquid.

\begin{figure}[tbph]
\centering
\includegraphics[width=0.7\linewidth]{Figures/gerda-view.png}
\caption{Diagram of the GERDA detector. Liquid argon is also inside the copper shielding with the detectors submerged.}
\label{fig:gerda-labelled}
\end{figure}

\begin{comment}
\subsubsection*{EXO-200}
The Enriched Xenon Observatory (EXO) is an experiment searching for \zeronubb~ in 80.6\% enriched $^{136}$Xe \cite{Albert:2014awa}. The Xe is used as a liquid time projection chamber (TPC) that can record the spatial components of ionization signals. This experiment benefits from the self-shielding of Xe, as the $\gamma$ attenuation length can be much smaller than the size of the detector itself.

\begin{figure}[htpb]
\centering
\includegraphics[width=0.7\linewidth]{Figures/EXO.png}
\caption[Cutaway view of the EXO-200 experimental apparatus.]{Cutaway view of the EXO-200 experimental apparatus. Figure from \cite{Auger:2012gs}}
\label{fig:jinst1205p05010}
\end{figure}
\end{comment}

\subsubsection*{NEMO-3}
The Neutrino Ettore Majorana Observatory (NEMO-3) collected \twonubb~ data from multiple isotopes including $^{100}$Mo, $^{82}$Se, $^{130}$Te, $^{116}$Cd, $^{150}$Nd, $^{96}$Zr, and $^{48}$Ca \cite{Bongrand:2011ei}. NEMO-3 also searched for \zeronubb~ in $^{100}$Mo, $^{82}$Se, $^{48}$Ca, and $^{150}$Nd \cite{Bongrand:2011ei}\cite{::2016dpe}\cite{Arnold:2016ezh}. Unlike the other experiments listed in this section, NEMO-3 did not utilize a detector = source method, instead using separate detectors and sources. The experimental apparatus included a tracking chamber and a calorimeter to measure the electrons produced in double beta decay.

\begin{figure}[htbp]
\centering
\includegraphics[width=0.7\linewidth]{Figures/nemo3_3930_42373_top_2}
\caption{A \zeronubb~ candidate event from NEMO-3. The electrons are emitted from a nucleus in the source foil and deposit energy in the calorimeter after leaving tracks in the tracking chamber.}
\label{fig:nemo3393042373top2}
\end{figure}


%\subsubsection*{Majorana}
%\subsubsection*{Heidelburg-Moscow}
\subsubsection*{KamLAND-Zen}
The Kamioka Liquid Scintillator Antineutrino Detector Zero neutrino double beta decay search (KamLAND-Zen) searches for \zeronubb~ in 90.8\% enriched $^{136}$Xe \cite{KamLAND-Zen:2016pfg}. The detector uses the Xe as a liquid scintillator acting as both the source and detector of \zeronubb.

\begin{figure}[tbph]
\centering
\includegraphics[width=0.7\linewidth]{Figures/KamlandZen}
\caption{A schematic of the Kamland-Zen experimental apparatus. The Xe liquid scintillator is held within the balloon at the center. Figure from \cite{::2015uaa}.}
\label{fig:kamlandzen}
\end{figure}

\begin{comment}
\begin{table}[htbp]
\centering
\begin{tabular}{|c|c|c|c|}
\hline 
Experiment & Isotope & \zeronubb~ Half-Life (yr) & Q-Value (keV) \\ 
\hline 
Gerda & $^{76}$Ge & $>5.2 \times 10^{25}$ & 2039 \\ 
\hline 
EXO-200 & $^{136}$Xe & $>1.1 \times 10^{25}$ & 2458 \\ 
\hline 
NEMO &  &  \\ 
\hline 
Majorana &  &  \\ 
\hline 
Heidelburg-Moscow &  &  \\ 
\hline 
Kamland-Zen &  &  \\ 
\hline 
CUORE &  & 2528 \\ 
\hline 
\end{tabular} 
\end{table}
\end{comment}

\section{CUORE Experimental Apparatus}

CUORE (Cryogenic Underground Observatory for Rare Events) is located in Gran Sasso, Italy, and utilizes a bolometric method to search for \zeronubb. The experiment will run with 988 crystals of TeO$_2$ held at roughly 10 mK. The source of \zeronubb~are the $^{130}$Te nuclei inside of the crystals ($34.2\%$ natural abundance \cite{Fehr200483}). Thus, in this setup, the crystals act as both sources and detectors of \zeronubb~When energy is deposited in the crystals, such as from the electrons emitted during \zeronubb, the temperature of the crystals rises. The Q-value of the decay of interest, $^{130}\textrm{Te} \rightarrow ^{130}\textrm{Xe} + e + e$, is $2527.518\pm 0.013$ keV \cite{Redshaw:2009cf}\cite{Scielzo:2009nh}\cite{Rahaman:2011zz}. This high Q-value is well-separated from other naturally occuring environmental $\gamma$'s except for $^{60}$Co at 2506 keV and $^{208}$Tl at 2615 keV. However, the resolution of the CUORE crystals will be $\sim5$ keV, which means that these peaks will not affect the region of interest, but have other uses that will be discussed more in sections (\color{green}Section Names\color{black}).

\subsection{CUORE Detectors}

The CUORE detectors are 988 $ 5\times 5 \times 5~ \textrm{cm}^3$ crystals arranged into 19 towers with 13 floors containing 4 crystals each as shown in \hyperref[fig:cuore-detector-array0]{Fig. \ref*{fig:cuore-detector-array0}}. The total detector mass is 741 kg with 206 kg of $^{130}$Te. 

\begin{figure}[htbp]
\centering
\includegraphics[width=0.7\linewidth]{"Figures/CUORE detector array_0"}
\caption{A rendering of the CUORE towers. The 19 towers are arranged into rows of 3, 4, 5, 4, and 3 towers.}
\label{fig:cuore-detector-array0}
\end{figure}


The crystals in each tower are held by copper frames, with polytetrafluoroethylene (PTFE) in between the copper and the crystals. The crystals respond to energy deposition by a small temperature rise according to
\begin{equation}
\delta T = E/C(T).
\end{equation}
At low C, $\delta T$ is maximized; therefore, the crystals are operated at low temperatures where C(T) is given by the Debye Model:
\begin{equation}
C(T)\sim T^3.
\end{equation}
At 10 mK, the operating temperature of CUORE, the heat capacity is $\sim2$ nJ/K, which causes a 1 MeV energy deposit to have a $\delta T$ of $\sim0.1$ mK.

This temperature rise is measured by a neutron-transmutation-doped (NTD) germanium thermistors which are glued onto each crystal. These thermistors allow the temperature change of the crystals to be read out by electronics as a change in applied voltage. Also glued on the crystals are silicon resistors which are used as a Joule heater. Bonded to the thermistors and the heaters are gold wires which are bonded to copper tape with a polyethylene naphthalate (PEN) substrate (collectively called Cu-PEN) on wire trays along the sides of the towers, shown in \hyperref[fig:bondedchips]{Fig. \ref*{fig:bondedchips}}.
\begin{figure}[htbp]
\centering
\includegraphics[width=0.7\linewidth]{Figures/BondedChips}
\caption{A germanium NTD on the crystal with four gold wires that have been bonded to both the NTD and to copper pads on the Cu-PEN tape.}
\label{fig:bondedchips}
\end{figure}
These signals are then read out by room-temperature electronics on top of the cryostat. A rendering of a full tower with the copper frames, wire trays, PTFE, and crystals can be seen at \hyperref[fig:SingleTowerWithZoom]{Fig. \ref*{fig:SingleTowerWithZoom}}

\begin{figure}[htbp]
\centering
\includegraphics[width=0.7\linewidth]{Figures/C0TowerZoom.pdf}
\caption[A rendering of a single tower in the CUORE detector assembly with a cutout for a single floor. Adjacent floors share the same frames.]{A rendering of a single tower in the CUORE detector assembly with a cutout for a single floor. Adjacent floors share the same frames. Diagram from \cite{Alduino:2016vjd}}
\label{fig:SingleTowerWithZoom}
\end{figure}

After an energy deposition causes the temperature of the crystal to rise, the crystal returns slowly back to the baseline temperature due to the weak thermal connection between the crystals and the copper frames which are held at 10 mK. A diagram of the thermal connection is shown in \hyperref[fig:thermal_crystal_cartoon]{Fig. \ref*{fig:thermal_crystal_cartoon}}. The rise and fall times of the signal are around 0.05 seconds and 0.2 seconds, respectively, for a 2615 keV signal.

(\color{yellow}More detail can be found in \cite{Alduino:2016vjd}\color{black})

\begin{figure}[htbp]
\centering
\includegraphics[width=0.7\linewidth]{Figures/BolDetector_Cartoon.pdf}
\caption[A diagram of the thermal connection of the TeO$_2$ crystals to the thermal bath provided by the copper frames. The weak thermal coupling is provided by the PTFE and the gold wires for the NTD and the heater. The thermistor is not shown to scale.]{A diagram of the thermal connection of the TeO$_2$ crystals to the thermal bath provided by the copper frames. The weak thermal coupling is provided by the PTFE and the gold wires for the NTD and the heater. The thermistor is not shown to scale. Diagram from \cite{Alduino:2016vjd}}
\label{fig:thermal_crystal_cartoon}
\end{figure}



\subsection{CUORE Cryostat}

Since the heat capacity of the CUORE crystals is so strongly dependent on temperature, it is important to have a system that can cool down the crystals to low temperatures and keep them running in stable conditions over long periods. To this end, we constructed a custom cryostat to house the crystals. In addition to cooling the crystals, the cryostat also houses the shielding for the crystals and needs to be made of radiopure materials, especially near the detectors themselves.

\begin{figure}[htbp]
\centering
\includegraphics[width=\linewidth]{Figures/Cryostat_Adjusted.png}
\caption{A CAD drawing of a cutaway the CUORE cryostat showing the internal shielding of the cryostat. The temperature stages of the cryostat are also shown. Not included are the external shields which extend around the sides of the cryostat}
\label{fig:cryostat_cad_cutout}
\end{figure}


\subsubsection*{Cooling Systems}
\begin{itemize}
\item FCS
\item Pulse Tubes
\item Dilution Fridge

%The CUORE cryostat 

\end{itemize}

\subsubsection*{Radiopurity}

\begin{figure}[htbp]
\centering
\includegraphics[width=0.7\linewidth]{Figures/CUORE_background_budget}
\caption{CUORE background budget. The CUORE goal is 0.01 counts/keV/kg/y with most of the background due to the surfaces of cryostat elements nearest to the crystal and the bulk of the Roman lead.}
\label{fig:istogramma}
\end{figure}

The background goal of CUORE is to have no greater than 0.01 counts/keV/kg/yr in the region of interest, defined to be a 100 keV interval from 2470 keV to 2570 keV. In order to achieve this low background, the parts nearest to the detectors undergo intensive cleaning. The contamination of these surfaces is given in \hyperref[tab:NearDetectorSources]{Table \ref*{tab:NearDetectorSources}} (\textcolor{blue}{citation needed}).

\begin{table}[htbp]
\centering
\caption{The $^{232}$Th and $^{238}$U bulk contamination of sources near the detector.}
\label{tab:NearDetectorSources}
\begin{tabular}{|c|c|c|}
\hline 
 Component & $^{232}$Th [g/g] & $^{238}$ U [g/g] \\ 
\hline 
TeO$_2$ crystals & $< 2.1\times 10^{-13}$ & $<5.3\times 10^{-14}$ \\ 
\hline 
NOSV copper & $<5.0 \times 10^{-13}$ & $<5.3 \times 10^{-12}$ \\ 
\hline 
NTD sensors & $< 1.0 \times 10^{-9}$ & $<1.0 \times 10^{-9}$ \\ 
\hline 
Bonding gold wires & $< 1.0 \times 10^{-8}$ & $<1.0 \times 10^{-9}$ \\ 
\hline 
Si heaters &   $<8\times 10^{-11}$ & $<1.7 \times 10^{-10}$ \\ 
\hline 
PTFE holders & $<1.5\times 10^{-12}$ & $<1.8 \times 10^{-12}$ \\ 
\hline 
Cu-PEN cables & $<4.4\times 10^{-10}$ & $1.1 \times 10^{-10}$ \\ 
\hline
Glue & $<2.2\times 10^{-10}$ & $<8.2\times10^{-10}$ \\
\hline 
\end{tabular} 
\end{table}

\subsection{CUORE Calibration}

With the stringent background requirements for the CUORE detectors, (\color{red}Calculate event rate for bkg events?\color{black}) , there are not enough events to calibrate the detectors to understand their energy response. The Detector Calibration System (DCS) was designed in response to this need. This system deploys known calibration sources to the detectors and irradiates them so that the energy peaks from these sources can be identified. 


\section{CUORE Simulations}

Using physics simulations can be a valuable tool for understanding a detector (\color{blue}citation needed\color{black}).  CUORE utilizes Geant4, a Monte Carlo simulation package which specializes in simulating the physics of particle propagation through matter \cite{Allison:2006ve}\cite{Agostinelli:2002hh}. For the simulations in CUORE, we implement a code called \textit{qshields} which is written for Geant4.9.6.p03 and uses the Livermore physics list. The CUORE simulation is used to generate particles on the surfaces and inside the volumes of different detector, cryostat, and shielding components. In the \textit{qshields} simulation, particular attention is taken to recreating the geometry of components on or near the detector. Volume elements further away that lack direct line of sight to the crystals and have lead and copper shielding between them and the detectors are simplified in the \textit{qshields} geometry. This was done in order both to simplify the geometry and to remove degenerate volumes. The CUORE MC geometry in \textit{qshields} can be seen in \hyperref[fig:cuorecryostatmc]{Fig. \ref*{fig:cuorecryostatmc}}.

\begin{figure}[htbp]
\centering
\includegraphics[width=0.7\linewidth]{Figures/CUORE_cryostat_MC}
\caption{Cutaway of the CUORE cryostat in \textit{qshields}. Not cutaway are the calibration source strings and the inner calibration tubes. The external shielding is not shown. Compare with \hyperref[fig:cryostat_cad_cutout]{Fig. \ref*{fig:cryostat_cad_cutout}} for the CAD drawing of the same cryostat.}
\label{fig:cuorecryostatmc}
\end{figure}

As \textit{qshields} does not take into account the detector response, we apply the measured parameters of the detectors to the simulation, e.g. energy thresholds, energy resolution, and pileup. Also applied is a calculation of how many crystals register an energy deposit during a small time window ($\pm5$ ms). \zeronubb events, if they exists, would generally deposit energy in only a single crystal, which is an important analysis cut to be used in CUORE. This information, called multiplicity, is also important tool for understanding how different types of physics events will deposit energy in multiple crystals and is one of the main advantages of the segmented detector approach used in CUORE.

\subsection{\twonubb~ in Simulations}
\textit{qshields} also plays a large role in the \twonubb~ analysis. As the \twonubb~ signal is a spectrum up to the 2528 keV Q-value for $^{130}$Te in CUORE, there are many background sources to understand and recreate. With input from radioactivity measurements of the components of CUORE, such as from \hyperref[tab:NearDetectorSources]{Fig. \ref*{tab:NearDetectorSources}}, a background model can be recreated that reproduces the observed spectrum. Statistical analysis can then be used to determine the best-fit \twonubb signal with this spectrum.
  
\section{Location of Work}

This work will take place at both Yale University and Gran Sasso National Lab in the Abruzzo region of Italy. 


\section{Dissertation Organization Proposal}

\begin{itemize}
\item Introduction \checkmark
\item Theory \checkmark
\item CUORE and other experiments (Kamland, Majorana, EXO, Heidelburg, GERDA, NEMO)
\item Field of 2nuBB \checkmark 
\item CUORE detector apparatus
\item Physics Reach of CUORE
\item Simulation for CUORE \checkmark
\item Analysis \checkmark
\item timeline
\item conclusion/summary
\end{itemize}

\section{Timetable}

\newpage

\begin{comment}
\begin{thebibliography}{8}
\bibitem{Higgs-ATLAS}
ATLAS Collaboration. Phys. Lett. B \textbf{716} (2012) 1
\bibitem{Higgs-CMS}
CMS Collaboration. Phys. Lett. B \textbf{716} (2012) 30
\bibitem{Homestake Neutrino}
R. Davis, D. Harmer, K. Hoffman. Phys. Rev. Lett. \textbf{20}, 1205 (1968)
\bibitem{Super-K} Super-Kamiokande Collaboration. Phys. Rev. Lett. \textbf{81}, 1562 (1998)
\bibitem{Sudbury Neutrino}
SNO Collaboration. Phys. Rev. Lett. \textbf{87}, 071301 (2001)
\bibitem{Sakharov}
A. D. Sakharov. Pisma Zh. Eksp. Teor. Fiz., \textbf{5}:32-35, 1967
\bibitem{zeronubb_PhaseSpace}
J. Barea, J. Kotila, F. Iachello. Phys. Rev. C \textbf{87} (2013) 014315
\bibitem{Sensitivity Calculation}
F. Alessandra, R. Ardito, D. R. Artusa et. al. (2011) p. 14 \url{arXiv:1109.0494}  \textcolor{blue}{check link}
\bibitem{Te130 Abundance}
M. A. Fehr, M. Rehk{\"a}mper, A. N. Halliday. Int. J. Mass Spectom. \textbf{232}, 83 (2004)
\bibitem{QValue_1}
M. Redshaw, B.J. Mount, E.G. Myers, and F.T. Avignone, III. Phys. Rev. Lett. \textbf{102} (2009) 212502
\bibitem{QValue_2}
N.D. Scielzo et al., Phys. Rev. C \textbf{80} (2009) 025501
\bibitem{QValue_3}
S. Rahaman et al. Phys. Lett. B \textbf{703} (2011) 412
\bibitem{0nu_LongAnalysisPaper}
CUORE Collaboration. Eur. Phys. J. C \textbf{74}, 2956 (2014)
\bibitem{CUORE_DetectorPaper}
C. Alduino et al 2016 \textit{JINST} \textbf{11} P07009
\bibitem{Geant4_1}
J. Allision, et al. \textit{IEEE Transactions on Nuclear Science} \textbf{53}, 270-278 (2006)
\bibitem{Geant4_2}
S. Agostinelli, et al. \textit{Nuclear Instruments and Methods A} \textbf{50}, 250-303 (2003)

\end{thebibliography}
\end{comment}

\bibliography{Bibliography}
\bibliographystyle{ieeetr}

\end{document}